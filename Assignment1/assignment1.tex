\documentclass{article}
\usepackage{graphicx} % Required for inserting images
\usepackage{booktabs} % For better table formatting
\usepackage{longtable} % If the table is long and spans multiple pages
\usepackage{listings}
\usepackage{xcolor}
\title{\textbf{Datamining 1: Assignment 1, }}
\author{Group 17\\Simon Pislar, Robin Forslund, Edvin Illes, \\Vidar Pellving, Filip Bodlund Trostén}
\date{September 2024}
\begin{document}

\maketitle

\section{Task 1}
\subsection*{a.}
The data type of the assigned attribute 'id' is an integer.

\subsection*{b.}
When choosing integers we sacrifice the abilitiy to easialy identify a single entry. 
For the ease of comparing the diffrent entries. 
Due to integer comparing being a lot faster the string comparing.

\subsection*{c.}
Average: -5.705507692307693
Standard deviation: 303.7655790655846

\section{Task 2}
\subsection*{Virginica}
3000 instances
\subsection*{Setosa}
3000 instances
\subsection*{Versicolor}
500 instances

\section{Task 3}
\subsection*{a.}
Missing values could cause faulty interperation of the result. In our case it made the average length negative. 
This is of course impossible.

\subsection*{b.}
Average: 3.5275947028025865
Standard deviation: 2.102330347553885
\subsection*{c.}
Average: 3.519642582036666
Standard deviation: 2.018227888523091

\subsection*{d.}
Since the outlier in this case was 22 times the standard deviation.
We can assume that it was noise since the probability for that being a correct observation is effectivily 0.

\subsection*{e.}
We would firstly handle the missing data. 
To know if something is an outlier we need to check the standard deviation and if data is skwed from missing data this will be very difficult.

\subsection*{f.}
No we would not remove outliers in this case. In this specific scenario, it's hard to imagine what an outlier would be in a practical sense.
Lets say we find a record in this social network that has a deviation of 4, heuristicly this could be a famous actor (assuming the social network is follower based).
Now lets assume our definition of an outlier is the record with the highest degree centrality and deviation of 22. The probability for this as discussed earlier was 0\%
which could indicate a record/person with \"botted\" followers. The implication however of removing a fale positive is of greater consequence when dealing with "real" human data
so a default answer should be no (unless we have a good system to remedy false mutations of (rollback) data in the database).

\section{Task 4}
\subsection*{a.}
Average: 0.43865202936075254
Standard deviation: 0.3255206271811437

\subsection*{b.}
Average: 7.005813447642338e-17, which is basically 0
Standard deviation: 1.0

\subsection*{c.}
Components selected: 2

\subsection*{d.}
Variance: 96.3%

\subsection*{e.}
How is the first component defined as a combination of the original attributes?

Let t(x) denote the transformation by MinMaxScaler(0,1) followed by StandardScaler(). Then the relation between the 
principal components PC 1 and PC 2 and the original variables pl, pw, sl, sw is:

PC 1 = 0.51996187 * t(pl) -0.29793284 * t(pw) + 0.57300707 * t(sl) + 0.55905149 * t(sw) \\
PC 2 = 0.38106173 * t(pl) + 0.91949846 * t(pw) + 0.05478206 * t(sl) + 0.07945728 * t(sw)



\subsection*{f.}
How many components would have been selected after 4.4 (that is, with an
attribute expressed on a larger range)?
Components selected: 1

\subsection*{g.}
How many components would have been selected after 4.5 (that is, with an
outlier)?
Components selected: 2


\section{Task 5}
\begin{table}[ht]
    \centering
    \caption{Summary of Sampling Methods and Specific Questions}
    \begin{tabular}{@{}lcccc@{}}
    \toprule
    \textbf{Question} & \textbf{Simple} & \textbf{Sampling} & \textbf{Bootstrapping} & \textbf{Stratified (5.3)} & \textbf{Stratified (5.4)} \\ \midrule
    Number of Iris Versicolor   & & & & & \\
    Number of Iris Setosa       & & & & & \\
    Number of Iris Virginica    & & & & & \\
    Are there repeated identifiers? & & & & & \\
    Does the number of Iris Versicolor included in the sample change if you change the local random seed? & Yes/No & Yes/No & Yes/No & Yes/No & Yes/No \\ \bottomrule
    \end{tabular}
    \label{tab:sampling_summary}
\end{table}

\end{document}